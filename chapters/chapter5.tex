\chapter{Analise Resultados}\label{chapter:analysis-result}
\section{Análise de objetivos}
Nesta secção, farei uma análise a todos os objetivos definidos no plano, às diferenças entre o trabalho realizado e o trabalho definido e o porquê dessas diferenças, conforme os objetivos definidos no plano \ref{fig:work_plan}.

\subsection{Acolhimento e introdução ao desenvolvimento}
Este foi o primeiro objetivo definido e realizado neste estágio. Tinha como finalidade introduzir-me às ferramentas e tecnologias utilizadas na empresa. Esta fase foi consideravelmente mais curta do que o planeado. Muitas destas ferramentas já eram-me conhecidas, e em algumas delas já tinha trabalhado anteriormente em projetos do curso e outras foram lecionadas em unidades curriculares. Refiro-me ao \textit{java}, ao \textit{svelte} e ao \textit{git}. Às restantes tecnologias adaptei-me rapidamente, e ao fim de dois dias já as conseguia utilizar para desenvolver.

\vspace{0.5cm}
\subsection{Levantamento de requisitos}
Esta fase, que totalizou 40 horas, foi fundamental para a minha integração no projeto, o que terminou num entendimento da arquitetura do sistema e das regras do negócio. Primeiro, procurei compreender as tecnologias e ferramentas utilizadas no projeto. Dado que a área do projeto onde trabalhei abrangia partes distintas, \textit{frontend} e \textit{backend}, foi essencial conhecer o código já desenvolvido pelos outros colaboradores da empresa, as diferenças entre as versões das tecnologias e a criação de novas ferramentas.

Apesar de ter sido uma fase demorada, em comparação com a anterior, foi aqui que aprendi como funcionava cada parte do projeto e a forma correta de implementar novas funcionalidades. 

\clearpage
\subsection{Escolher a API de Mapas}
Esta foi uma tarefa muito interessante. Nela consegui pesquisar e realizar uma análise sobre as tecnologias lecionadas numa das unidades curriculares de que mais gostei de realizar neste curso, sistemas de informação geográfica. Nesta análise verifiquei as funcionalidades de algumas das tecnologias disponíveis no mercado, comparei-as e selecionei a que se adequava melhor ao nosso projeto, com base nos requisitos definidos pela empresa. Após algumas horas de testes e pesquisa, cheguei a uma conclusão que se mostrou acertada no momento da sua implementação, a escolha do \textit{mapbox}.

Um aspeto que diferenciou-se um pouco foi a quantidade de horas previstas. Apesar de estarem previstas 120 horas, esta pesquisa foi realizada de forma mais rápida, devido à necessidade de apresentar os pontos na aplicação e ao facto de, em equipa, chegarmos a um consenso de que aquela tecnologia era a correta para o projeto.

\vspace{0.5cm}
\subsection{Desenvolvimento da UI}
Uma das etapas mais importantes deste estágio foi o desenvolvimento da \textit{\acs{ui}}. Neste objetivo, foi-me dada a responsabilidade de alterar a interface, efetuar melhorias de usabilidade e incorporar as funcionalidades dos mapas e gráficos. Estas foram as primeiras tarefas que realizei, a implementação do mapa teve a maior prioridade, onde procurei criar componentes reutilizáveis, elaborar documentação para outros programadores e definir formas de conversão de dados, entre outros aspetos.

Da mesma forma que criei estas funcionalidades, também alterei algumas que já existiam, como a adição de conteúdo às páginas, a modificação de alguns \textit{layouts} e a alteração do \textit{design system}.

Acredito que no fim desta etapa, que decorreu de forma intercalada com outras, as horas previstas foram cumpridas e, provavelmente, até ultrapassadas. Isto aconteceu pela complexidade da criação de alguns componentes e a refatoração da interface existe.

\vspace{0.5cm}
\subsection{Desenvolvimento dos serviços}
Nesta tarefa, desenvolvi alguns serviços \textit{RESTful} para aumentar as funcionalidades da interface e resolver problemas existentes nos mesmos. Em conjunto com os colaboradores da empresa, consegui implementar algumas funcionalidades que não previa implementar, como por exemplo, a adição de dados recebidos via \acs{udp} na base de dados.

O plano, em concreto, não continha qualquer tarefa no espectro que me competia, pelo que tive alguma liberdade, concedida pelo orientador, para verificar alguns serviços que não funcionavam, analisar o porquê de não funcionarem e corrigi-los, como por exemplo, o serviço de email. A forma como abordei a criação de funcionalidades também foi essencial para manter a segurança e o bom funcionamento do servidor.

Nesta fase, uma vez que realizei algumas tarefas que não estavam completamente planeadas, parte do tempo das tarefas iniciais foi realocado a esta.

\vspace{0.5cm}
\subsection{Testes funcionais}
Esta foi a única tarefa que não se realizou durante o estágio. Apesar de os testes serem uma parte importante para verificar as funcionalidades de um sistema, durante este estágio, não houve oportunidade nem o momento certo para os realizar. Apesar de não os ter realizado, outras tarefas foram levadas a cabo durante as horas previstas.

\vspace{0.5cm}
\subsection{Relatório}
Esta foi uma pequena tarefa realizada durante o último dia de estágio. Nesta tarefa, foram criadas as bases onde este relatório foi elaborado.

\section{Considerações finais}
Esta análise revela um percurso de estágio que, embora preso ao plano inicial, demonstrou uma capacidade de adaptação às necessidades reais e aos imprevistos ocorridos no projeto. Verifica-se que, para além das tarefas planeadas, como a análise de \textit{\acs{api}s} de georreferenciação e o desenvolvimento da interface, parte do tempo foi dedicada à resolução de problemas e à implementação de melhorias. A decisão de realocar algumas das horas destinadas a outras tarefas demonstrou a necessidade de concretizar funcionalidades para o sistema.

Deste modo, o trabalho realizado ao longo das 466 horas de estágio resultou num contributo mais robusto e alinhado com os desafios práticos encontrados.