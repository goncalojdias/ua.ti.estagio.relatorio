\chapter{Introdução}%
\label{chapter:introduction}

O presente relatório pretende descrever as atividades desenvolvidas durante o Estágio
inserido no 3º ano da Licenciatura em Tecnologias da Informaçao, lecionado na 
Escola Superior de Tecnologia e Gestão de Águeda coordenado
pelo Prof. Ivan Pires.

A realização do estágio tem com objetivo colocar em prática todos os conhecimentos,
teóricos e práticos adquiridos e desenvolvidos nas unidades curriculares.
O estágio decorreu na Wiseware Solutions, Gafanha da Encarnação, entre 11
de Fevereiro e 23 de Maio de 2025 sobre a orientação de
Gustavo Corrente.

Os principais objetivos do estágio foram obter uma maior experiência em contexto
real de trabalho, adquirir novos conhecimentos sobre uma estrutura organizacional de
empresas e desenvolver competências práticas e teóricas a nível profissional.

O presente relatório encontra-se dividido em 4 partes. Na primeira parte é feita uma
brevedescrição da Entidadede Acolhimento, os trabalhos nela feitos e como está dividida.
Na segunda parte, é feita uma análise à rede informática da empresa, no qual é possível
observar um diagrama lógico da rede e alguns problemas que ela possuiu. Nessa mesma
fase são também feitas propostas de alteração para a mesma rede.

Na segunda parte pretende-se falar sobre o trabalho principal do estágio, onde o ob-
jetivo era criar uma Mini rede empresarial informática com serviços fundamentais para
o seu funcionamento, como por exemplo, um cloud storage, servidor web, DNS, DHCP,
VPN, entre outros. 
A quarta e última parte pretende falar de um programa criado para
ajudar os funcionários da empresa a fazer o seu trabalho mais rapidamente.
Para finalizar apresentam-se as considerações finais que abrangem uma reflexão final à
cerca das competências adquiridas e um balanço final do estágio.

\begin{introduction}
\end{introduction}