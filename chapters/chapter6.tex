\chapter{Conclusões}\label{chapter:conclusion}
O estágio curricular ao serviço da Wiseware foi uma experiência única e gratificante que permitiu aprofundar conhecimentos teóricos, técnicos e adquirir competências profissionais.
\newline

Ao concluir este estágio, sinto-me capaz de realizar diversas funções que permitem-me ter sucesso dentro de um ambiente empresarial, especialmente em funções dedicadas à criação de interfaces, no desenvolvimento de \acs{api}s, encontrar novas tecnologias e trabalhar em equipa.
\newline

Durante a realização deste estágio, consegui executar muitas das tarefas que estavam propostas no plano, assim como realizei outras que não constavam no mesmo.

Por vezes, consegui realizar as tarefas num tempo inferior ao proposto. No entanto, houve também tarefas desafiadoras onde dediquei algum tempo à leitura da documentação, na compreensão das suas nuances e na definição da melhor estratégia para a sua implementação.
\newline

Algo que se revelou desafiador, tanto a nível profissional como pessoal, foi a alteração de mentalidade que tive de realizar devido às diferenças entre o trabalho académico e o trabalho profissional, onde é necessária maior agilidade e usabilidade na produção, pois não somos os únicos a desenvolver, avaliar e utilizar o código.
\newline

O desenvolvimento da interface foi a área de maior aprendizagem, onde o contacto direto com os utilizadores e a incorporação do seu \textit{feedback} se revelaram cruciais para compreender as suas necessidades e entregar um produto com maior valor.
\newline

Juntamente com o desenvolvimento da interface, a criação de serviços \textit{RESTful} permitiu a aplicação de vários conhecimentos lecionados no curso, bem como de outros adquiridos neste estágio, onde tive de os aprender e resolver os problemas que foram surgindo pelo caminho.
\newline

Face aos vários desafios encontrados ao longo da realização do estágio, existiram momentos em que foi necessário solicitar ajuda ao orientador e também muitas vezes realizar pesquisas de como resolver determinados problemas e como corrigi-los corretamente.
\newline

% FIM
Por fim é possível dizer que este estágio foi bastante positivo, tanto profissionalmente como pessoalmente, pois deu-me a vontade de aprender sobre novos temas e iniciar novos projetos. Juntamente com os novos conhecimentos, que virão a ser importantes em projetos futuros, aprendi também como é trabalhar em um ambiente empresarial e que pouco tempo nesse ambiente é muito importante para a formação profissional.