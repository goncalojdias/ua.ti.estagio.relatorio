\chapter{Plano de trabalho}%
\label{chapter:work-plan}

Antes do estágio começar, foi definido um plano de trabalho que estabelecia os principais objetivos a atingir no decorrer do mesmo, \autoref{fig:work_plan}. Embora este plano represente a base para a realização do estágio, é expectável que, no final, possam existir variações entre o trabalho inicialmente planeado e o efetivamente realizado.

O documento apresenta os objetivos propostos, as atividades a desenvolver para a sua concretização e as horas prevista por atividade. Estes objetivos encontram-se organizados por áreas de trabalho distintas, conforme ilustrado na referida \autoref{fig:work_plan}.

A fase inicial de acolhimento e introdução ao desenvolvimento, com uma previsão de 40 horas, visou a introdução com o ambiente de trabalho e, a aquisição de conhecimentos sobre as tecnologias e ferramentas essenciais usadas pela empresa no processo de desenvolvimento, como \textit{Git}, \textit{Gitlab}, \textit{Jenkins}, o ambiente \textit{Java} e \textit{Svelte}.

Seguiu-se o levantamento de requisitos, também estimado em 40 horas. Este período dedicou-se à aquisição de uma compreensão aprofundada do funcionamento do projeto \textit{aniposture}, nomeadamente dos seus objetivos, estrutura e tecnologias envolvidas.

O estudo e seleção da \textit{\acs{api}} de Mapas constituiu um objetivo crucial, com 120 horas alocadas. Esta tarefa implicou uma análise comparativa das funcionalidades, custos e relação custo-benefício de diversas \acs{api}s de georreferenciação, como \textit{Google Maps} e \textit{Mapbox}, para suportar a tomada de decisão.

Com 120 horas previstas, o desenvolvimento da interface de utilizador, \acs{ui}, centrou-se na implementação gráfica e na integração da \acs{api} de georreferenciação previamente selecionada.

Paralelamente, o desenvolvimento de serviços, com igual carga horária de 120 horas, focou-se na criação de serviços \textit{RESTful} e na implementação de \textit{endpoints} essenciais para o funcionamento do \textit{backend} da aplicação.

A realização de testes funcionais, com uma duração estimada de 24 horas, visou validar o correto funcionamento da aplicação e a sua conformidade com os requisitos definidos.

Por fim, dedicou-se tempo à finalização da escrita do relatório, com uma previsão de 2 horas, ainda no decorrer do período de estágio.
\newline

Este plano de trabalho serviu, portanto, como um guia para a realização do estágio, ao definir as principais tarefas e as suas respetivas metas. A sua existencia foi fundamental para direcionar os esforços, servir como base de comparação e permitir medir o progresso do estágio.

% Antigo

% \begin{enumerate}
%     \item O primeiro objetivo correspondeu à fase de \textit{Acolhimento e introdução ao desenvolvimento}. Esta etapa teve como principal finalidade a minha introdução com as tecnologias e ferramentas utilizadas pela empresa no processo de desenvolvimento.
%     \item O segundo objetivo foi o \textit{Levantamento de requisitos}. Nesta fase, a missão consistiu em adquirir um conhecimento aprofundado sobre o funcionamento do projeto \textit{aniposture}, incluindo os seus objetivos, estrutura, tecnologias envolvidas, entre outros aspetos.
%     \item O terceiro objetivo foi a \textit{Escolher a API de Mapas}. Esta etapa teve como finalidade a realização de um estudo e comparação de funcionalidades, preços e relação custo-beneficio das diferentes \textit{APIs} de georreferenciação. 
%     \item O quarto objetivo correspondeu ao \textit{Desenvolvimento da \acs{ui}}. Esta fase visava a implementação da interface gráfica integrando a \textit{API} de georreferenciação previamente selecionada. 
%     \item O quinto objetivo consistiu o \textit{Desenvolvimento de serviços}. Este objetivo procurava  o desenvolvimento de serviços RESTful e a implementação de endpoints úteis para  o funcionamento do backend da aplicação. 
%     \item O sexto objetivo foi \textit{Testes funcionais}. Esta etapa teve como principal finalidade a realização de testes funcionais de forma a verificar o bom funcionamento da aplicação.
%     \item O setimo e último objetivo foi o \textit{Relatório}. Ainda durante o período de estágio, foram alocadas algumas horas para a redação e finalização do presente relatório.
% \end{enumerate}


