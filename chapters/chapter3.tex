\chapter{Plano de trabalho}%
\label{chapter:work-plan}

Antes do estágio começar, foi definido um plano de trabalho que estabelecia os principais objetivos a atingir no decorrer do mesmo, \autoref{fig:work_plan}. Apesar de constituir a base para a a realização do estágio, é natural que, no final, possam existir algumas variações entre o trabalho realizado e o trabalho inicialmente planeado.

No plano de trabalho foram definidos vários objetivos, bem como o número de horas previstas para a sua concretização. Estes objetivos foram organizados por áreas de trabalho distintas.

\begin{enumerate}
    \item O primeiro objetivo correspondeu à fase de \textit{Acolhimento e introdução ao desenvolvimento}. Esta etapa teve como principal finalidade a minha introdução com as tecnologias e ferramentas utilizadas pela empresa no processo de desenvolvimento.
    \item O segundo objetivo foi o \textit{Levantamento de requisitos}. Nesta fase, a missão consistiu em adquirir um conhecimento aprofundado sobre o funcionamento do projeto \textit{aniposture}, incluindo os seus objetivos, estrutura, tecnologias envolvidas, entre outros aspetos.
    \item O terceiro objetivo foi a \textit{Escolher a API de Mapas}. Esta etapa teve como finalidade a realização de um estudo e comparação de funcionalidades, preços e relação custo-beneficio das diferentes \textit{APIs} de georreferenciação. 
    \item O quarto objetivo correspondeu ao \textit{Desenvolvimento da \acs{ui}}. Esta fase visava a implementação da interface gráfica integrando a \textit{API} de georreferenciação previamente selecionada. 
    \item O quinto objetivo consistiu o \textit{Desenvolvimento de serviços}. Este objetivo procurava  o desenvolvimento de serviços RESTful e a implementação de endpoints úteis para  o funcionamento do backend da aplicação. 
    \item O sexto objetivo foi \textit{Testes funcionais}. Esta etapa teve como principal finalidade a realização de testes funcionais de forma a verificar o bom funcionamento da aplicação.
    \item O setimo e último objetivo foi o \textit{Relatório}. Ainda durante o período de estágio, foram alocadas algumas horas para a redação e finalização do presente relatório.
\end{enumerate}


